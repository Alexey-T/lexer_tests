% !TeX program = lualatex
% !TeX encoding = utf8

% The preamble is the first section of an input file, 
% before the text of the document itself, in which you tell 
\documentclass[12pt, letterpaper, twoside]{article}
\usepackage[utf8]{inputenc} %codification of the document
\usepackage[demo]{graphicx}
\usepackage{luacode, pythontex} 

\title{\LaTeX Lexer for Synwrite/CudaText}
\begin{document}
\maketitle

\section*{Introduction}

\section{Lists}

The unordered (unnumbered) lists are produced by the itemize environment. Each entry must be preceded by the control sequence \string\item.

\begin{itemize}
  \item The individual entries are indicated with a black dot, a so-called bullet.
  \item The text in the entries may be of any length.
\end{itemize}

\section{Mathematics}

\LaTeX allows two writing modes for mathematical expressions: the inline mode and the display mode. 

The well known Pythagorean theorem \(x^2 + y^2 = z^2\) was 
proved to be invalid for other exponents. 
Meaning the next equation has no integer solutions:
 
\[ x^n + y^n = z^n \]

The mass-energy equivalence is described by the famous equation
 
$$E=mc^2$$
 
discovered in 1905 by Albert Einstein. 
In natural units ($c$ = 1), the formula expresses the identity
 
\begin{equation}\label{key}
	E=mc^2
\end{equation}

\section{Figures and tables}

\subsection{Figures}
If we want to further specify how \LaTeX should include our image in the document (length, height, etc), we can pass those settings in the following format:
 
\begin{figure}[h!]
 \centering
 \includegraphics[draft, width=0.5\textwidth]{example-image-b}
 \caption{Example of a parametric plot ($\sin (x), \cos(x), x$)}
\end{figure}

\subsection{Tables}
Below you can see the simplest working example of a table:
\begin{center}
\begin{tabular}{ |c|c|c| } 
 \hline
 cell1 & cell2 & cell3 \\ 
 cell4 & cell5 & cell6 \\ 
 cell7 & cell8 & cell9 \\ 
 \hline
\end{tabular}
\end{center}

\section{The verbatim text}
The default tool to display code in \LaTeX is verbatim, which generates an output in monospaced font.

For example:\\
\verb|Text inside \verb is printed directly and all \LaTeX{} commands are ignored.|

And also standard verbatim environment do the same:
\begin{verbatim}
Text enclosed inside verbatim environment 
is printed directly and all \LaTeX{} commands are ignored.
\end{verbatim}

\section{Using Lua\LaTeX}

The first job is, as ever, to have the Lua function at the ready
\begin{luacode}
   -- 'mchoose' is patterned after the posting in http://stackoverflow.com/a/15302448.
   -- Thanks, @egreg, for providing the pointer to this posting!
   function mchoose( n, k )
     if (k == 0 or k == n) then
       return 1 
     else 
       return (n * mchoose(n - 1, k - 1)) / k 
     end
   end

   -- print an arbitrary length integer as formatted string
   function fwrite(z) 
      tex.sprint ( string.format("%.0f",z) )
   end
\end{luacode}

\section{Using Python}

\begin{pycode}

import numpy as np
import matplotlib.pyplot as plt
def f(t):
    return np.exp(-t) * np.cos(2*np.pi*t)

t1 = np.arange(0.0, 5.0, 0.1)
t2 = np.arange(0.0, 5.0, 0.02)

plt.figure(1)
plt.subplot(211)
plt.plot(t1, f(t1), 'bo', t2, f(t2), 'k')

plt.subplot(212)
plt.plot(t2, np.cos(2*np.pi*t2), 'r--')
plt.savefig('myplot.pdf', bbox_inches='tight')
plt.show()
print (r'\includegraphics[width=\linewidth]{myplot.pdf}')

\end{pycode}

\begin{pycode}
def fib(n): # nth Fibonacci value
     a, b = 0, 1
     for i in range(n):
         a, b = b, a + b
     return a
\end{pycode}

\end{document}
